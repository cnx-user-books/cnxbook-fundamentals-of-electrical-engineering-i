\documentclass[12pt]{article}
\usepackage{times,epsfig,amsmath}
\input{newcommand}
\addtolength{\oddsidemargin}{-.75in}
\addtolength{\evensidemargin}{-.75in}
\addtolength{\textwidth}{1.3in}
\addtolength{\topmargin}{-.9in}
\addtolength{\textheight}{1.5in}
%
\newcounter{quiz}
\setcounter{quiz}{2}
\renewcommand{\thequiz}{\arabic{quiz}}
%
% Change long hypen appearance
%
\def\hlinefill{\leaders\hrule height3pt depth-2.5pt\hfill}
\def\emrule{\thinspace\hbox to .75em{\hlinefill}\thinspace}
%
\makeatletter
%
% Set path for EPSFIG
%
%\define@key{Gin}{figure}{\def\Gfigname{:Figures:#1}}
%\define@key{Gin}{file}{\def\Gfigname{:Figures:#1}}
%
% Problem environment
%
\newcounter{problem}
\renewcommand{\theproblem}{Q\thequiz.\arabic{problem}}
\newcounter{problempart}
\renewcommand{\theproblempart}{\alph{problempart}}
\newcounter{problemsubpart}
\renewcommand{\theproblemsubpart}{\roman{problemsubpart}}
\newenvironment{problems}%
{
\begin{list}%
{\bf\theproblem\hfill}%
{\usecounter{problem}\setlength{\itemindent}{-2em}\setlength{\labelwidth}{0em}}
}%
{\end{list}}
%
\newenvironment{problemparts}%
{\begin{list}%
{\bf(\theproblempart)\hfil}{\usecounter{problempart}}
}%
{\end{list}}
%
\newenvironment{problemsubparts}%
{\begin{list}%
{(\theproblemsubpart)\hfil}{\usecounter{problemsubpart}}
}%
{\end{list}}
\begin{document}
\sloppy
\begin{center}
\large\textbf{Electrical Engineering 241\\
Quiz \Roman{quiz}\\
October 24, 2002}
\end{center}
\par\noindent
One and one-half hour exam.
Two 8 1/2"$\times$11" information sheet can be used.
Each problem is given equal weight.
Please sign the pledge when you are finished.
%
\begin{problems}
\item \textbf{Pre-emphasis or De-emphasis?}\\
In audio applications, prior to analog-to-digital conversion signals are passed through what is known as a \emph{pre-emphasis circuit} that leaves the low frequencies alone but provides increasing gain at increasingly higher frequencies beyond some frequency $f_0$.
\emph{De-emphasis circuits} do the opposite and are applied after digital-to-analog conversion.
After pre-emphasis, digitization, conversion back to analog and de-emphasis, the signal's spectrum should be what it was.

The following op-amp circuit has been designed for pre-emphasis or de-emphasis (Samanatha can't recall which).
\par\noindent\centerline{\epsfig{figure=opamp21.eps}}
\begin{problemparts}
\item
Is this a pre-emphasis or de-emphasis circuit?
Find the frequency $f_0$ that defines the transition from low to high frequencies.
\item
What is the circuit's output when the input voltage is $\sin(2\pi f t)$, with $f=4\textrm{ kHz}$?
\item
What circuit could perform the opposite function to your answer for the first part?
\end{problemparts}
%***************
\item \textbf{A Different Sampling Scheme}\\
A signal processing engineer from Texas A\&M claims to have developed an improved sampling scheme.
He multiplies the bandlimited signal by the depicted periodic pulse signal to perform sampling.
\par\noindent\centerline{\epsfig{figure=sig47.eps}}
\begin{problemparts}
\item
Find the Fourier spectrum of this signal.
\item
Will this scheme work?
If so, how should $\samplingint$ be related to the signal's bandwidth?
If not, why not.
\end{problemparts}
%*********
\item \textbf{Secret Communication}\\
An amplitude-modulated secret message $\msg(t)$ has the following form.
\[
\received(t) = \amplitude[1+\msg(t)]\cos \bigl(2\pi (f_c+f_o) t\bigr)
\]
The message signal has a bandwidth of $W$~Hz and a magnitude less than $1$ ($|\msg(t)|<1$).
The idea is to offset the carrier frequency by $f_o$~Hz from standard radio carrier frequencies.
Thus, ``off-the-shelf'' coherent demodulators would assume the carrier frequency has $f_c$~Hz.
Here, $f_o<W$.
\begin{problemparts}
\item
Sketch the spectrum of the demodulated signal produced by a coherent demodulator tuned to $f_c$~Hz.
\item
Will this demodulated signal be ``scrambled'' version of the original?
If so, how so;
if not, why not?
\item
Can you develop a receiver that can demodulate the message without knowing the offset frequency $f_o$?
\end{problemparts}
%************
\end{problems}
\end{document}