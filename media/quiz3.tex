\documentclass[12pt]{article}
\usepackage{times,epsfig,amsmath}
\input{newcommand}
\addtolength{\oddsidemargin}{-.75in}
\addtolength{\evensidemargin}{-.75in}
\addtolength{\textwidth}{1.3in}
\addtolength{\topmargin}{-.9in}
\addtolength{\textheight}{1.5in}
%
\newcounter{quiz}
\setcounter{quiz}{3}
\renewcommand{\thequiz}{\arabic{quiz}}
%
% Change long hypen appearance
%
\def\hlinefill{\leaders\hrule height3pt depth-2.5pt\hfill}
\def\emrule{\thinspace\hbox to .75em{\hlinefill}\thinspace}
%
\makeatletter
%
% Set path for EPSFIG
%
%\define@key{Gin}{figure}{\def\Gfigname{:Figures:#1}}
%\define@key{Gin}{file}{\def\Gfigname{:Figures:#1}}
%
% Problem environment
%
\newcounter{problem}
\renewcommand{\theproblem}{Q\thequiz.\arabic{problem}}
\newcounter{problempart}
\renewcommand{\theproblempart}{\alph{problempart}}
\newcounter{problemsubpart}
\renewcommand{\theproblemsubpart}{\roman{problemsubpart}}
\newenvironment{problems}%
{
\begin{list}%
{\bf\theproblem\hfill}%
{\usecounter{problem}\setlength{\itemindent}{-2em}\setlength{\labelwidth}{0em}}
}%
{\end{list}}
%
\newenvironment{problemparts}%
{\begin{list}%
{\bf(\theproblempart)\hfil}{\usecounter{problempart}}
}%
{\end{list}}
%
\newenvironment{problemsubparts}%
{\begin{list}%
{(\theproblemsubpart)\hfil}{\usecounter{problemsubpart}}
}%
{\end{list}}
\begin{document}
\sloppy
\begin{center}
\large\textbf{Electrical Engineering 241\\
Quiz \Roman{quiz}\\
November 21, 2002}
\end{center}
\par\noindent
One and one-half hour exam.
Three 8 1/2"$\times$11" information sheet can be used.
Each problem is given equal weight.
Please sign the pledge when you are finished.
%
\begin{problems}
\item \textbf{A Digital Filter}\\
A digital filter has the depicted unit-sample response.
\par\noindent
\centerline{\epsfig{figure=sig48.eps}}
\begin{problemparts}
\item
What is the difference equation that defines this filter's input-output relationship?
\item
What is this filter's transfer function?
\item
What is the filter's output when the input is $\sin(\pi n/4)$?
\end{problemparts}
%***************
\item \textbf{Digital Amplitude Modulation}\\
Two ELEC~241 students disagree about a homework problem.
The issue concerns the discrete-time signal $\signal(n)\cos 2\pi f_0n$,
where the signal $\signal(n)$ has no special characteristics and the modulation frequency $f_0$ is known.
Sammy says that he can recover $\signal(n)$ from its amplitude-modulated version by the same approach used in analog communications.
Samantha says that approach won't work.
\begin{problemparts}
\item
What is the spectrum of the modulated signal?
\item
Who is correct? Why?
\item
The teaching assistant does not want to take sides.
He tells them that if $\signal(n)\cos2\pi f_0n$ and $\signal(n)\sin 2\pi f_0n$ were both available, $\signal(n)$ can be recovered.
What does he have in mind?
\end{problemparts}
%*********
\newpage
\item \textbf{Digital Stereo}\\
Just as with analog communication, it should be possible to send two signals simultaneously over a digital channel.
Assume you have two CD-quality signals (each sampled at 44.1~kHz with 16~bits/sample).
One suggested transmission scheme is to use a quadrature BPSK scheme.
If $\bit^{(1)}(n)$ and $\bit^{(2)}(n)$ each represent a bit stream, the transmitted signal has the form
\[
\xmit(t) = \amplitude\sum_n \left[\bit^{(1)}(n)\sin 2\pi f_c (t-n\period)\pulse(t-n\period) + \bit^{(2)}(n)\cos 2\pi f_c (t-n\period)\pulse(t-n\period)\right]
\]
where $\pulse(t)$ is a unit-amplitude pulse having duration $\period$ and $\bit^{(1)}(n)$, $\bit^{(2)}(n)$ equal either $+1$ or $-1$ according to the bit being transmitted for each signal.
The channel adds white noise and attenuates the transmitted signal.
\begin{problemparts}
\item
What value would you choose for the carrier frequency $f_c$?
\item
What is the transmission bandwidth?
\item
What receiver would you design that would yield both bit streams?
\end{problemparts}
%************
\end{problems}
\end{document}